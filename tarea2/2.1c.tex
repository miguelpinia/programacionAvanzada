\begin{proof}[Solución]
Para demostrar la correctez del código anterior, tendremos que demostrar:
\newline
$\{a=b \wedge a=b \} \hspace*{1cm}  m:=a \hspace*{.5cm} \{m=a \} \hspace*{.5cm} (1)$   
\newline
$\{a=b \wedge \neg (a=b) \} \hspace*{.5cm}  m:=b \hspace*{.5cm} \{m=a \} \hspace*{.5cm} (2)$   
\vspace*{0.5cm}
\newline
P.D (1) ie $ \{a=b \wedge a=b \} \hspace*{0.3cm}  m:=a \hspace*{.3cm} \{m=a \}$   
\vspace*{0.5cm}
\newline
Demostración
\vspace*{0.3cm}
\newline
Sabemos que:
\newline
Dado $\{P\} V:=E \{Q\}$
\newline
$\Rightarrow  \{P\} = \{Q_{E}^{V} \} = \{V=E, Q \} $
\vspace*{0.3cm}
\newline
Usando la definición anterior, tenemos que:
Si $ \{a=b \wedge a=b \} \hspace*{0.3cm}  m:=a \hspace*{.3cm} \{m=a \}$  
\newline
$ \{ P \} = \{ (m=a)_{a}^{m}   \} = \{m=a,(m=a)\} $
\newline
$ \hspace*{0.8cm} =\{ a=a \}$
\newline
$ \hspace*{0.8cm} = \{ \} $
\newline
$ \hspace*{0.8cm} \Rightarrow  \{ \} \hspace*{0.3cm} m:=a \hspace*{0.3cm} \{ m=a \} $
\newline
Sabemos que $ \{ \} $ fortalece cualquier aseveración.
\vspace*{0.3cm}
\newline
Por lo tanto tenemos:
\newline
$ \{a=b \wedge a=b \} \hspace*{0.3cm} \Rightarrow  \cancel{\{ \}}$
\newline
$ \hspace*{0.8cm} \Rightarrow  \cancel{\{ \}} \hspace*{0.3cm} m:=a \hspace*{0.3cm} \{ m=a \} $
\newline
\noindent\rule{6cm}{0.4pt}
\newline
$ \{a=b \wedge a=b \} \hspace*{0.3cm}  m:=a \hspace*{.3cm} \{m=a \}$
\vspace*{0.3cm}
\newline
Por lo tanto (1) es válida
\vspace*{0.5cm}
\newline
P.D (2) ie $\{a=b \wedge \neg (a=b) \} \hspace*{.5cm}  m:=b \hspace*{.5cm} \{m=a \}  $ 
\newline
Sabemos que:
\newline
Dado $\{P\} V:=E \{Q\}$
\newline
$\Rightarrow  \{P\} = \{Q_{E}^{V} \} = \{V=E, Q \} $
\vspace*{0.3cm}
\newline
Usando la definición anterior, tenemos que:
Si $\{a=b \wedge \neg (a=b) \} \hspace*{.5cm}  m:=b \hspace*{.5cm} \{m=a \} $  
\newline
$ \{ P \} = \{ (m=a)_{b}^{m}   \} = \{m=a,(m=a)\} $
\newline
$ \hspace*{0.8cm} = \{ b=a \} $
\newline
$ \hspace*{0.8cm} \Rightarrow  \{ b=a \} \hspace*{0.3cm} m:=b \hspace*{0.3cm} \{ m=a \} $
\vspace*{0.3cm}
\newline
Pero como $P \wedge \neg P \Rightarrow false $ y nosotros tenemos: $a=b \wedge \neg (a=b) $
\newline 
Y sabemos que \emph{false} implica cualquier aseveración, podemos concluir que (2) es verdadera.
\vspace*{0.3cm}
\newline
Como (1) y (2) son ciertas, tenemos entonces:
\vspace*{0.3cm}
\newline
$ \{a=b \wedge a=b \} \hspace*{1cm}  m:=a \hspace*{.5cm} \{m=a \} $
\newline
$ \{a=b \wedge \neg (a=b) \} \hspace*{.5cm}  m:=b \hspace*{.5cm} \{m=a \} $
\newline
\noindent\rule{10.5cm}{0.4pt}
\newline
$ \{a = b \} \hspace*{0.3cm} if \hspace*{0.3cm} a == b \hspace*{0.3cm} then \hspace*{0.3cm} m := a \hspace*{0.3cm} else \hspace*{0.3cm} 
m := b \hspace*{0.3cm} \{m := a \}  $

\end{proof}