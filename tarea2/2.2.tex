\begin{proof}[Solución]

  Encontremos el invariante primero. Sean $Z_n$ y $W_n$ los valores de
  \textit{Z} y \textit{W} en la iteración $n \ge 0$ dentro del ciclo
  while. Veamos la evolución de $Z_N$ y $W_n$ a través del ciclo:

  En la iteración 0 tenemos:

  \begin{equation*}
    \begin{split}
      Z_0 = X,\\W_o = Y
    \end{split}
  \end{equation*}

  En la iteración 1:

  \begin{equation*}
    \begin{split}
      Z_1 = Z_0 - 1 = X - 1\\W_1 = W_0 - 1 = Y - 1
    \end{split}
  \end{equation*}

  En la iteración 2:

  \begin{equation*}
    \begin{split}
      Z_2 = Z_1 - 1 = X - 2\\W_2 = W_1 -  1 = Y - 2
    \end{split}
  \end{equation*}

  Y en la iteración n-ésima:

  \begin{equation}
    \label{Zn1}
    \begin{split}
      Z_n = Z_{n-1} - 1 = X - n, \\W_n = W_{n-1} - 1 = Y - n
    \end{split}
  \end{equation}

  De \ref{Zn1} tenemos que:

  \begin{equation}
    n = Y - W_n \Rightarrow Z_n = X - n = X - (Y - W_n)
  \end{equation}

  P.D. $Z_n = X  - (Y - W_n)$ es invariante.

  \textbf{Caso base: } n = 0\\
  \begin{equation*}
    Z_0 = X - n = X - (Y - W_0) = X - (Y - Y) = X
  \end{equation*}

  Se cumple que $Z_0 = X$\\
  \textbf{Hipótesis de inducción:} Supongamos que en el paso n-ésimo
  se cumple que:

  \begin{equation}
    \begin{split}
        Z_n = X - n = X - (Y - W_n) \\
        n = Y - W_n
    \end{split}
  \end{equation}

  \textbf{Paso Inductivo:} P.D. que en el paso n + 1 es verdadera la
  relación.

  En la iteración n + 1, tenemos que:

  \begin{equation*}
    \begin{split}
      Z_{n+1} = X - (n + 1)\\
      W_{n+1} = Y - (n + 1)
    \end{split}
  \end{equation*}

  Entonces:

  \begin{equation*}
    \begin{split}
      Z_{n + 1} & = X - (n + 1)\\
      & = X - n - 1\\
      & = X - (Y - W_n) - 1\\
      & = X - Y + W_n - 1\\
      & = X - Y + Y - n - 1\\
      & = X - Y + Y - (n + 1)\\
      & = X - Y + W_{n + 1}\\
      & = X - (Y - W_{n + 1})\\
    \end{split}
  \end{equation*}

  Por lo tanto se cumple el paso inductivo y por el principio de
  inducción que: $Z_n = X - n = X - (Y - W_n)$

  \textbf{Este procedimiento lo que hace es calcular la diferencia $X
    - Y$, en la iteración 0, Z vale X y cuando la iteración está en n
    = Y, la rutina regresa $Z = X - Y$, terminando el ciclo y la
    ejecución.}

\end{proof}