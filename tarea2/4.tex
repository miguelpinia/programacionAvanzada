\begin{proof}[Solución]

  Para demostrar la correctez del código anterior, tenemos que
  demostrar usando la regla de inferencia de la asignación, sxoe cumple
  la precondición del programa.

  Entonces hay que determinar que se puede llegar a \{P\} = $\{n \ge
  1\}$.

  Entonces:

  \begin{equation}
    \label{hoare}
    \{P\} = \{Q\}^V_E = \{w \ge 2^{n-1}\}^w_{calcula(n)} = \{calcula(n)
    \ge 2 ^{n - 1}\}
  \end{equation}

  Ahora hay que demostrar el \textit{if} de la función calcula,
  demostrando las ternas de Hoare \ref{if1} y \ref{if2}:

  \begin{equation}
    \label{if1}
    \{c = 1 \land (n == 0 \lor n == 1)\} \hspace{2em} return 1 \hspace{2em} \{c == n!\}
  \end{equation}

  \begin{equation}
    \label{if2}
   \{c = 1 \land \neg (n == 0 \lor n == 1)\} \hspace{2em} for(i = 2; i \leq n;
   n++)\hspace{1em} c = c * i \hspace{2em}\{c = n!\}
  \end{equation}

  P.D. \ref{if1} es válida. Demostración\\

  Sabemos que:

  Dado $\{P\} V:=E \{Q\} \Rightarrow \{P\} = \{Q\}^V_E = \{V=E, Q\}$

  Usando la definición anterior tenemos que si:

  \begin{equation*}
    \{c = 1 \land (n == 0 \lor n == 1)\} \hspace{2em} \text{return}\hspace{1em} c
    \hspace{2em} \{c = n!\}
  \end{equation*}

  entonces:

  \begin{equation*}
    \begin{split}
      \{p\} &= \{c = n!\}^c_1\\
      & = \{1 = n!\}\\
      & = \{1 = 1! \lor 1 = 0!\}\\
      & = \{\}\\
      & \Rightarrow \{\} \hspace{2em} \text{return}\hspace{1em} c\hspace{2em} \{c == n!\}\\
    \end{split}
  \end{equation*}

  Sabemos que $\{\}$ fortalece cualquier aseveración, por lo tanto
  tenemos:

  \begin{equation*}
    \begin{split}
      \{c = 1 \land \neg (n == 0 \lor n == 1)\} \Rightarrow \cancel{\{\}}\\
      \Rightarrow \cancel{\{\}} \hspace{1em} \text{return}\hspace{1em}
      c\hspace{1em} \{c == n!\}\\
      \rule{6cm}{0.4pt}\\
      \{c = 1 \land \neg (n == 0 \lor n == 1)\}\hspace{1em} \text{return}\hspace{1em}
      c\hspace{1em} \{c == n!\}\\
    \end{split}
  \end{equation*}

  Por lo tanto \ref{if1} es válida.\\

  P.D. \ref{if2} es válida. Demostración:\\

    Sabemos que:

  Dado $\{P\} V:=E \{Q\} \Rightarrow \{P\} = \{Q\}^V_E = \{V=E, Q\}$

  Usando la definición anterior tenemos que si:

  \begin{equation*}
    \{c = 1 \land \neg (n == 0 \lor n == 1)\} \hspace{1em} for(i = 2; i \leq n;
    n++)\hspace{1em} c = c * i \hspace{1em}\{c = n!\}
  \end{equation*}

  Entonces, la ejecución del ciclo \textit{for}, la podemos reescribir
  como c = n!, dado que va multiplicando i por su subsecuente hasta n.

  Entonces:

  \begin{equation*}
    \begin{split}
      \{p\} &= \{c = n!\}^c_{n!}\\
      & = \{n! = n!\}\\
      & = \{\}\\
      & \Rightarrow \{\} \hspace{1em} for(i = 2; i \leq n;
    n++)\hspace{1em} c = c * i \hspace{1em} \{c == n!\}\\
    \end{split}
  \end{equation*}


    Sabemos que $\{\}$ fortalece cualquier aseveración, por lo tanto
  tenemos:

  \begin{equation*}
    \begin{split}
      \{c = 1 \land (n == 0 \lor n == 1)\} \Rightarrow \cancel{\{\}}\\
      \Rightarrow \cancel{\{\}}\hspace{1em} for(i = 2; i \leq n;
    n++)\hspace{1em} c = c * i \hspace{1em} \{c == n!\}\\
      \rule{6cm}{0.4pt}\\
      \{c = 1 \land (n == 0 \lor n == 1)\}\hspace{1em} for(i = 2; i \leq n;
      n++)\hspace{1em} c = c * i \hspace{1em} \{c == n!\}\\
    \end{split}
  \end{equation*}

  Por lo tanto \ref{if2} es válida.\\

  Retomando \ref{hoare}, tenemos que:

  \begin{equation}
    \{P\} = \{Q\}^V_E = \{w \ge 2^{n-1}\}^w_{calcula(n)} = \{calcula(n)
    \ge 2 ^{n - 1}\} = {n! \ge 2^{n - 1}}
  \end{equation}

  Entonces, hay que probar por inducción que $n! \ge 2^{n - 1}$ para
  $n \ge 1$ para que por regla de fortalecimiento tengamos que $\{n
  \ge 1\} \Rightarrow \{n! \ge 2^{n - 1}\}$.

  \textbf{P.D.} $n! \ge 2^{n - 1}$ para $n \ge 1$\\
  \textbf{Demostración:}\\
  \textbf{Caso base:}

\end{proof}