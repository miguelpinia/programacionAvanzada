\documentclass[11pt]{article}
\usepackage[utf8]{inputenc}
\usepackage[spanish]{babel}
\usepackage{cancel}
\usepackage{amsmath}
\usepackage{amsfonts}
\usepackage{amssymb}
\usepackage{amsthm}
\usepackage{makeidx}
\usepackage{graphicx}
\usepackage{verbatim}
\usepackage{algorithm}
\usepackage{mathtools}
\usepackage{algpseudocode}
\usepackage{array, tabularx}
\usepackage[left=3.2cm,right=3.2cm,top=3.2cm,bottom=3cm]{geometry}
\usepackage{fancyhdr}
\graphicspath{ {./figures/} }
\makeatletter
\renewcommand{\ALG@name}{Algoritmo}
\makeatother
\renewcommand{\qedsymbol}{$\blacksquare$}
\DeclarePairedDelimiter\ceil{\lceil}{\rceil}
\DeclarePairedDelimiter\floor{\lfloor}{\rfloor}

\author{Miguel Angel Piña Avelino\\Cinthia Rodríguez Maya}
\date{\today}
\title{Tarea 2\\Programación Avanzanda}
\fancyhf{}
\rhead{Programación Avanzada}
\lhead{Tarea 2}
\rfoot{Página \thepage}

\begin{document}
\maketitle

\begin{enumerate}

\item Demostrar que el siguiente código es correcto:
\begin{verbatim}
{a = b} if a == b then m := a else m := b {m := a}
\end{verbatim}

\input{1}

\item Explicar que hacen las siguientes rutinas, encontrando el
  invariante y demostrar que es válido por inducción matemática.

  \begin{algorithm}
    \caption{}
    \begin{algorithmic}
      \Procedure{COMP}{X, Y; Z}
      \State $Z \leftarrow X$
      \State $W \leftarrow Y$
      \While{$W > 0$}
      \State $Z \leftarrow Z + Y$
      \State $W \leftarrow W - 1$
      \EndWhile
      \State \textbf{return} Z
      \EndProcedure
    \end{algorithmic}
  \end{algorithm}

  % Aquí se hace la llamada al archivo 2.1.tex y sólo se compila este archivo.
    \begin{proof}[Solución]
    Encontremos el invariante primero. Sean $Z_n$ y $W_n$ los valores
    de \textit{Z} y \textit{W} en la iteración $n \ge 0$ dentro del
    ciclo while. Veamos la evolución de $Z_n$ y $W_n$ a través del
    ciclo:

    En la iteración 0 tenemos:

    \begin{equation*}
      \begin{split}
        Z_0 = X,\\ W_0= Y
      \end{split}
    \end{equation*}



    En la iteración 1:

    \begin{equation*}
      \begin{split}
        Z_1 = Z_0 + Y = X + Y,\\W_1 = W_0 - 1 = Y - 1
      \end{split}
    \end{equation*}

    En la iteración 2:

    \begin{equation*}
      \begin{split}
        Z_2 = Z_1 + Y = X + 2Y,\\W_2 = W_1 - 1 = Y - 2
      \end{split}
    \end{equation*}

    y en la iteración n:

    \begin{equation}
      \label{Zn}
      \begin{split}
        Z_n = Z_{n-1} + Y = X + nY,\\W_n = W_{n-1} - 1 = Y - n
      \end{split}
    \end{equation}

    De \ref{Zn} tenemos que:

    \begin{equation}
        n = Y - W_n \Rightarrow Z_n = X + Y(Y - W_n) = X + Y^2 - YW_n
    \end{equation}

    P.D. $Z_n = X + Y^2 - YW_n$ es invariante.

    \textbf{Caso base:} n = 0\\
    \begin{equation*}
      Z_0 = X + Y^2 - YW_0 = X + Y^2 - YY = X + Y^2 - Y^2 = X
    \end{equation*}

    Se cumple que $Z_0 = X$\\
    \textbf{Hipótesis de inducción:} Supongamos que en el paso n-ésimo
    se cumple que:

    \begin{equation}
      Z_n = X + Y^2-YW_n
    \end{equation}

   \textbf{Paso Inductivo:} P.D. que en el paso n + 1 es verdadera la
   relación.

   En la iteración n + 1, tenemos que:

   \begin{equation*}
     \begin{split}
       Z_{n+1} = X + (n + 1)Y\\
       W_{n+1} = Y - (n + 1)
     \end{split}
   \end{equation*}

   Entonces:

   \begin{equation*}
     \begin{split}
       Z_{n+1} & = X + (n + 1)Y = X + nY + Y \\
       & = X + Y(Y - W_n) + Y \hspace{2cm}\ldots \text{usando el hecho que n = } Y
       - W_n\\
       & = X + Y^2 - YW_n + Y\\
       & = X + Y^2 - Y(W_n - 1)\\
       & = X + Y^2 - Y(Y - n - 1) \hspace{1.5cm}\ldots W_n = Y - n\\
       & = X + Y^2 - Y(Y - (n + 1))\\
       & = X + Y^2 - YW_{n + 1} \hspace{2.5cm}\ldots W_{n+1} = Y - (n + 1)
     \end{split}
   \end{equation*}

   Por lo tanto se cumple el paso inductivo y por el principio de
   inducción $Z_n = X + nY = X + Y(Y - W_n) = X + Y^2 - YW_n$.

   \textbf{Este procedimiento lo que hace es calcular $X + Y^2$, en la
     iteración 0, Z vale X y cuando la iteración n = Y,  la rutina
     regresa $Z = X + Y^2$, terminando el ciclo y la ejecución}
  \end{proof}

  \begin{algorithm}
    \caption{}
    \begin{algorithmic}
      \Procedure{DIFF}{X, Y; Z}
      \State $Z \leftarrow X$
      \State $W \leftarrow Y$
      \While{$W > 0$}
      \State $Z \leftarrow Z - 1$
      \State $W \leftarrow W - 1$
      \EndWhile
      \State \textbf{return} Z
      \EndProcedure
    \end{algorithmic}
  \end{algorithm}

  \begin{proof}[Solución]

  Encontremos el invariante primero. Sean $Z_n$ y $W_n$ los valores de
  \textit{Z} y \textit{W} en la iteración $n \ge 0$ dentro del ciclo
  while. Veamos la evolución de $Z_N$ y $W_n$ a través del ciclo:

  En la iteración 0 tenemos:

  \begin{equation*}
    \begin{split}
      Z_0 = X,\\W_o = Y
    \end{split}
  \end{equation*}

  En la iteración 1:

  \begin{equation*}
    \begin{split}
      Z_1 = Z_0 - 1 = X - 1\\W_1 = W_0 - 1 = Y - 1
    \end{split}
  \end{equation*}

  En la iteración 2:

  \begin{equation*}
    \begin{split}
      Z_2 = Z_1 - 1 = X - 2\\W_2 = W_1 -  1 = Y - 2
    \end{split}
  \end{equation*}

  Y en la iteración n-ésima:

  \begin{equation}
    \label{Zn1}
    \begin{split}
      Z_n = Z_{n-1} - 1 = X - n, \\W_n = W_{n-1} - 1 = Y - n
    \end{split}
  \end{equation}

  De \ref{Zn1} tenemos que:

  \begin{equation}
    n = Y - W_n \Rightarrow Z_n = X - n = X - (Y - W_n)
  \end{equation}

  P.D. $Z_n = X  - (Y - W_n)$ es invariante.

  \textbf{Caso base: } n = 0\\
  \begin{equation*}
    Z_0 = X - n = X - (Y - W_0) = X - (Y - Y) = X
  \end{equation*}

  Se cumple que $Z_0 = X$\\
  \textbf{Hipótesis de inducción:} Supongamos que en el paso n-ésimo
  se cumple que:

  \begin{equation}
    \begin{split}
        Z_n = X - n = X - (Y - W_n) \\
        n = Y - W_n
    \end{split}
  \end{equation}

  \textbf{Paso Inductivo:} P.D. que en el paso n + 1 es verdadera la
  relación.

  En la iteración n + 1, tenemos que:

  \begin{equation*}
    \begin{split}
      Z_{n+1} = X - (n + 1)\\
      W_{n+1} = Y - (n + 1)
    \end{split}
  \end{equation*}

  Entonces:

  \begin{equation*}
    \begin{split}
      Z_{n + 1} & = X - (n + 1)\\
      & = X - n - 1\\
      & = X - (Y - W_n) - 1\\
      & = X - Y + W_n - 1\\
      & = X - Y + Y - n - 1\\
      & = X - Y + Y - (n + 1)\\
      & = X - Y + W_{n + 1}\\
      & = X - (Y - W_{n + 1})\\
    \end{split}
  \end{equation*}

  Por lo tanto se cumple el paso inductivo y por el principio de
  inducción que: $Z_n = X - n = X - (Y - W_n)$

  \textbf{Este procedimiento lo que hace es calcular la diferencia $X
    - Y$, en la iteración 0, Z vale X y cuando la iteración está en n
    = Y, la rutina regresa $Z = X - Y$, terminando el ciclo y la
    ejecución.}

\end{proof}

\item Demuestre si la siguiente terna de Hoare es válida:


\begin{verbatim}
{k > n} if (k < n) k := n else n:= k {n = k}
\end{verbatim}

\begin{proof}[Solución]
Para demostrar la correctez del código anterior, tendremos que demostrar:
\newline
$\{(k>n) \wedge (k<n) \} \hspace*{0.5cm}  k:=n \hspace*{.5cm} \{n=k \} \hspace*{.5cm} (1)$   
\newline
$\{k>n \wedge \neg (k<n) \} \hspace*{0.5cm}  n:=k \hspace*{.5cm} \{n=k \} \hspace*{.5cm} (2)$   
\vspace*{0.5cm}
\newline
P.D (1) ie $ \{(k>n) \wedge (k<n) \} \hspace*{0.3cm}  k:=n \hspace*{.5cm} \{n=k \} $   
\vspace*{0.5cm}
\newline
Demostración
\vspace*{0.3cm}
\newline
Sabemos que:
\newline
Dado $\{P\} V:=E \{Q\}$
\newline
$\Rightarrow  \{P\} = \{Q_{E}^{V} \} = \{V=E, Q \} $
\vspace*{0.3cm}
\newline
Usando la definición anterior, tenemos que:
Si $ \{(k>n) \wedge (k<n) \} \hspace*{0.5cm}  k:=n \hspace*{.5cm} \{n=k\} $  
\newline
$ \{ P \} = \{ (n=k)_{n}^{k} \} = \{k=n,(n=k)\} $
\newline
$ \hspace*{0.8cm} =\{ n=n \}$
\newline
$ \hspace*{0.8cm} = \{ \} $
\newline
$ \hspace*{0.8cm} \Rightarrow  \{ \} \hspace*{0.3cm} k:=n \hspace*{.5cm} \{n=k\} $
\newline
Sabemos que $ \{ \} $ fortalece cualquier aseveración.
\vspace*{0.3cm}
\newline
Por lo tanto tenemos:
\newline
$ \{(k>n) \wedge (k<n) \} \hspace*{0.3cm} \Rightarrow  \cancel{\{ \}}$
\newline
$ \hspace*{0.8cm} \cancel{\{ \}} \hspace*{0.3cm} k:=n \hspace*{.5cm} \{n=k\}$
\newline
\noindent\rule{6.5cm}{0.4pt}
\newline
$ \{(k>n) \wedge (k<n) \} \hspace*{0.3cm}  k:=n \hspace*{0.3cm} \{n=k \}$
\vspace*{0.3cm}
\newline
Por lo tanto (1) es válida
\vspace*{0.5cm}
\newline


P.D (2) ie $\{k>n \wedge \neg (k<n) \} \hspace*{.5cm}  n:=k \hspace*{.5cm} \{n=k \} $ 
\vspace*{0.5cm}
\newline
Demostración
\vspace*{0.3cm}
\newline
Sabemos que:
\newline
Dado $\{P\} V:=E \{Q\}$
\newline
$\Rightarrow  \{P\} = \{Q_{E}^{V} \} = \{V=E, Q \} $
\vspace*{0.3cm}
\newline
Usando la definición anterior, tenemos que:
Si $ \{k>n \wedge \neg (k<n) \} \hspace*{.5cm}  n:=k \hspace*{.5cm} \{n=k \} $  
\newline
$ \{ P \} = \{ (n=k)_{k}^{n} \} = \{n=k,(n=k)\} $
\newline
$ \hspace*{0.8cm} =\{ k=k \}$
\newline
$ \hspace*{0.8cm} = \{ \} $
\newline
$ \hspace*{0.8cm} \Rightarrow  \{ \} \hspace*{0.3cm} n:=k \hspace*{.5cm} \{n=k\} $
\newline
Sabemos que $ \{ \} $ fortalece cualquier aseveración.
\vspace*{0.3cm}
\newline
Por lo tanto tenemos:
\newline
$ \{(k>n) \wedge \neg(k<n) \} \hspace*{0.3cm} \Rightarrow  \cancel{\{ \}}$
\newline
$ \hspace*{0.8cm} \cancel{\{ \}} \hspace*{0.3cm} n:=k \hspace*{.5cm} \{n=k\}$
\newline
\noindent\rule{6.5cm}{0.4pt}
\newline
$ \{(k>n) \wedge \neg(k<n) \} \hspace*{0.3cm}  n:=k \hspace*{0.3cm} \{n=k \}$
\vspace*{0.3cm}
\newline
Por lo tanto (2) es válida
\newline
Como (1) y (2) son ciertas, tenemos entonces:
\vspace*{0.3cm}
\newline
$\{(k>n) \wedge (k<n) \} \hspace*{0.5cm}  k:=n \hspace*{.5cm} \{n=k \} $
\newline
$\{k>n \wedge \neg (k<n) \} \hspace*{.5cm}  n:=k \hspace*{.5cm} \{n=k \} $
\newline
\noindent\rule{9cm}{0.4pt}
\newline
$ \{k > n\} \hspace*{.3cm} if (k < n) \hspace*{.3cm} k := n \hspace*{.3cm} else \hspace*{.3cm} n:= k \hspace*{.3cm} \{n = k\} $
\end{proof}


\item Dada la siguiente función:
\begin{verbatim}
public static int calcula(int n) {
    int c = 1;
    if (n == 0 || n == 1)
        return 1;
    else
        for (int i = 2; i <= n; i++)
            c = c * i;
    return c;
}
\end{verbatim}
  Demuestre que la siguiente terna de Hoare es válida:

\begin{equation*}
\{n \ge 1\} \hspace{1cm} w:= calcula(n) \hspace{1cm} \{w \ge 2^{n - 1}\}
\end{equation*}

\begin{proof}[Solución]

  Para demostrar la correctez del código anterior, tenemos que
  demostrar usando la regla de inferencia de la asignación, sxoe cumple
  la precondición del programa.

  Entonces hay que determinar que se puede llegar a \{P\} = $\{n \ge
  1\}$.

  Entonces:

  \begin{equation}
    \label{hoare}
    \{P\} = \{Q\}^V_E = \{w \ge 2^{n-1}\}^w_{calcula(n)} = \{calcula(n)
    \ge 2 ^{n - 1}\}
  \end{equation}

  Por lo tanto $\{calcula(n) \ge 2^{n - 1}\} \hspace{1em} w:= calcula(n) \hspace{1em} \{w \ge 2^{n - 1}\}$

  Ahora hay que demostrar el \textit{if} de la función \textit{calcula},
  demostrando las ternas de Hoare \ref{if1} y \ref{if2}:

  \begin{equation}
    \label{if1}
    \{c = 1 \land (n == 0 \lor n == 1)\} \hspace{2em} return 1 \hspace{2em} \{c == n!\}
  \end{equation}

  \begin{equation}
    \label{if2}
   \{c = 1 \land \neg (n == 0 \lor n == 1)\} \hspace{2em} for(i = 2; i \leq n;
   n++)\hspace{1em} c = c * i \hspace{2em}\{c = n!\}
  \end{equation}

  \textbf{P.D}. \ref{if1} es válida.\\
  \textbf{Demostración}\\

  Sabemos que:

  Dado $\{P\} V:=E \{Q\} \Rightarrow \{P\} = \{Q\}^V_E = \{V=E, Q\}$

  Usando la definición anterior tenemos que si:

  \begin{equation*}
    \{c = 1 \land (n == 0 \lor n == 1)\} \hspace{2em} \text{return}\hspace{1em} c
    \hspace{2em} \{c = n!\}
  \end{equation*}

  entonces:

  \begin{equation*}
    \begin{split}
      \{p\} &= \{c = n!\}^c_1\\
      & = \{1 = n!\}\\
      & = \{1 = 1! \lor 1 = 0!\}\\
      & = \{\}\\
      & \Rightarrow \{\} \hspace{2em} \text{return}\hspace{1em} c\hspace{2em} \{c == n!\}\\
    \end{split}
  \end{equation*}

  Sabemos que $\{\}$ fortalece cualquier aseveración, por lo tanto
  tenemos:

  \begin{equation*}
    \begin{split}
      \{c = 1 \land \neg (n == 0 \lor n == 1)\} \Rightarrow \cancel{\{\}}\\
      \Rightarrow \cancel{\{\}} \hspace{1em} \text{return}\hspace{1em}
      c\hspace{1em} \{c == n!\}\\
      \rule{6cm}{0.4pt}\\
      \{c = 1 \land \neg (n == 0 \lor n == 1)\}\hspace{1em} \text{return}\hspace{1em}
      c\hspace{1em} \{c == n!\}\\
    \end{split}
  \end{equation*}

  Por lo tanto \ref{if1} es válida.\\

  \textbf{P.D.} \ref{if2} es válida.\\
  \textbf{Demostración:}\\

    Sabemos que:

  Dado $\{P\} V:=E \{Q\} \Rightarrow \{P\} = \{Q\}^V_E = \{V=E, Q\}$

  Usando la definición anterior tenemos que si:

  \begin{equation*}
    \{c = 1 \land \neg (n == 0 \lor n == 1)\} \hspace{1em} for(i = 2; i \leq n;
    n++)\hspace{1em} c = c * i \hspace{1em}\{c = n!\}
  \end{equation*}

  Entonces, la ejecución del ciclo \textit{for}, la podemos reescribir
  como c = n!, dado que va multiplicando i por su subsecuente hasta n.

  Entonces:

  \begin{equation*}
    \begin{split}
      \{p\} &= \{c = n!\}^c_{n!}\\
      & = \{n! = n!\}\\
      & = \{\}\\
      & \Rightarrow \{\} \hspace{1em} for(i = 2; i \leq n;
    n++)\hspace{1em} c = c * i \hspace{1em} \{c == n!\}\\
    \end{split}
  \end{equation*}


    Sabemos que $\{\}$ fortalece cualquier aseveración, por lo tanto
  tenemos:

  \begin{equation*}
    \begin{split}
      \{c = 1 \land (n == 0 \lor n == 1)\} \Rightarrow \cancel{\{\}}\\
      \Rightarrow \cancel{\{\}}\hspace{1em} for(i = 2; i \leq n;
    n++)\hspace{1em} c = c * i \hspace{1em} \{c == n!\}\\
      \rule{6cm}{0.4pt}\\
      \{c = 1 \land (n == 0 \lor n == 1)\}\hspace{1em} for(i = 2; i \leq n;
      n++)\hspace{1em} c = c * i \hspace{1em} \{c == n!\}\\
    \end{split}
  \end{equation*}

  Por lo tanto \ref{if2} es válida.\\

  Retomando \ref{hoare}, tenemos que:

  \begin{equation}
    \{P\} = \{Q\}^V_E = \{w \ge 2^{n-1}\}^w_{calcula(n)} = \{calcula(n)
    \ge 2 ^{n - 1}\} = {n! \ge 2^{n - 1}}
  \end{equation}

  Entonces, hay que probar por inducción que $n! \ge 2^{n - 1}$ para
  $n \ge 1$ para que por regla de fortalecimiento tengamos que $\{n
  \ge 1\} \Rightarrow \{n! \ge 2^{n - 1}\}$.

  \textbf{P.D.} $n! \ge 2^{n - 1}$ para $n \ge 1$\\
  \textbf{Demostración:}\\
  \textbf{Caso base:} n = 1
  $\Rightarrow 1! \ge 2^{1 - 1} \Rightarrow 1 \ge 2^0 = 1 \Rightarrow
  1 = 1$
  Por lo tanto se cumple el caso base.\\
  \textbf{Hipótesis de inducción:} Supongamos que se cumple para n:\\
  \begin{equation*}
    n! \ge 2^{n - 1}
  \end{equation*}
  \textbf{Paso inductivo}

  \begin{equation*}
    \begin{split}
      (n + 1)! &= n!(n + 1)\\
      & \ge 2^{n - 1}(n + 1) \hspace{2cm}\text{Por HI}\\
      & > 2^{n - 1}2\\
      & = 2^n\\
    \end{split}
  \end{equation*}

  Entonces se cumple que $(n + 1)! \ge 2^n$. Por lo tanto se cumple
  que $n! \ge 2^n$.

  Entonces,

    \begin{equation*}
      \begin{split}
        \{n \ge 1\} \Rightarrow \cancel{\{n! \ge 2^{n - 1}\}}\\
        \cancel{\{n! \ge 2^{n - 1}\}}\hspace{1em} w:= calcula(n) \hspace{1em} \{w \ge 2^{n - 1}\}\\
        \rule{6cm}{0.4pt}\\
        \{n \ge 1\}\hspace{1em} w:= calcula(n) \hspace{1em} \{w \ge 2^{n - 1}\}\\
    \end{split}
  \end{equation*}

  Por lo tanto la terna de Hoare $\{n \ge 1\}\hspace{1em} w:= calcula(n) \hspace{1em}
  \{w \ge 2^{n - 1}\}$ es correcta.

\end{proof}

\end{enumerate}

\end{document}
