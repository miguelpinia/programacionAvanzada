\documentclass[11pt]{article}
\usepackage[utf8]{inputenc}
\usepackage[spanish]{babel}
\usepackage{amsmath}
\usepackage{amsfonts}
\usepackage{amssymb}
\usepackage{amsthm}
\usepackage{makeidx}
\usepackage{graphicx}
\usepackage{verbatim}
\usepackage{algorithm}
\usepackage{mathtools}
\usepackage{algpseudocode}
\usepackage{array, tabularx}
\usepackage[left=3.2cm,right=3.2cm,top=3.2cm,bottom=3cm]{geometry}
\usepackage{fancyhdr}
\graphicspath{ {./figures/} }
\makeatletter
\renewcommand{\ALG@name}{Algoritmo}
\makeatother
\renewcommand{\qedsymbol}{$\blacksquare$}
\DeclarePairedDelimiter\ceil{\lceil}{\rceil}
\DeclarePairedDelimiter\floor{\lfloor}{\rfloor}

\author{Miguel Angel Piña Avelino\\Cinthia Rodríguez Maya}
\date{\today}
\title{Tarea 2\\Programación Avanzanda}
\fancyhf{}
\rhead{Programación Avanzada}
\lhead{Tarea 2}
\rfoot{Página \thepage}

\begin{document}
\maketitle

\begin{enumerate}

\item Demostrar que el siguiente código es correcto:
\begin{verbatim}
{a = b} if a == b then m := a else m := b {m := a}
\end{verbatim}

\item Explicar que hacen las siguientes rutinas, encontrando el
  invariante y demostrar que es válido por inducción matemática.

  \begin{algorithm}
    \caption{}
    \begin{algorithmic}
      \Procedure{COMP}{X, Y; Z}
      \State $Z \leftarrow X$
      \State $W \leftarrow Y$
      \While{W > 0}
      \State $Z \leftarrow Z + Y$
      \State $W \leftarrow W - 1$
      \EndWhile
      \State \textbf{return} Z
      \EndProcedure
    \end{algorithmic}
  \end{algorithm}

  \begin{proof}[Solución]
    Encontremos el invariante primero. Sean $Z_n$ y $W_n$ los valores
    de \textit{Z} y \textit{W} en la iteración $n \ge 0$ dentro del
    ciclo while. Veamos la evolución de $Z_n$ y $W_n$ a través del
    ciclo:

    En la iteración 0 tenemos:

    \begin{equation*}
      \begin{split}
        Z_0 = X,\\ W_0= Y
      \end{split}
    \end{equation*}



    En la iteración 1:

    \begin{equation*}
      \begin{split}
        Z_1 = Z_0 + Y = X + Y,\\W_1 = W_0 - 1 = Y - 1
      \end{split}
    \end{equation*}

    En la iteración 2:

    \begin{equation*}
      \begin{split}
        Z_2 = Z_1 + Y = X + 2Y,\\W_2 = W_1 - 1 = Y - 2
      \end{split}
    \end{equation*}

    y en la iteración n:

    \begin{equation}
      \label{Zn}
      \begin{split}
        Z_n = Z_{n-1} + Y = X + nY,\\W_n = W_{n-1} - 1 = Y - n
      \end{split}
    \end{equation}

    De \ref{Zn} tenemos que:

    \begin{equation}
        n = Y - W_n \Rightarrow Z_n = X + Y(Y - W_n) = X + Y^2 - YW_n
    \end{equation}

    P.D. $Z_n = X + Y^2 - YW_n$ es invariante.

    \textbf{Caso base:} n = 0\\
    \begin{equation*}
      Z_0 = X + Y^2 - YW_0 = X + Y^2 - YY = X + Y^2 - Y^2 = X
    \end{equation*}

    Se cumple que $Z_0 = X$\\
    \textbf{Hipótesis de inducción:} Supongamos que en el paso n-ésimo
    se cumple que:

    \begin{equation}
      Z_n = X + Y^2-YW_n
    \end{equation}

   \textbf{Paso Inductivo:} P.D. que en el paso n + 1 es verdadera la
   relación:

   En la iteración n + 1, tenemos que:

   \begin{equation*}
     \begin{split}
       Z_{n+1} = X + (n + 1)Y\\
       W_{n+1} = Y - (n + 1)
     \end{split}
   \end{equation*}

   Entonces:

   \begin{equation*}
     \begin{split}
       Z_{n+1} & = X + (n + 1)Y = X + nY + Y \\
       & = X + Y(Y - W_n) + Y \hspace{2cm}\ldots \text{usando el hecho que n = } Y
       - W_n\\
       & = X + Y^2 - YW_n + Y\\
       & = X + Y^2 - Y(W_n - 1)\\
       & = X + Y^2 - Y(Y - n - 1) \hspace{1.5cm}\ldots W_n = Y - n\\
       & = X + Y^2 - Y(Y - (n + 1))\\
       & = X + Y^2 - YW_{n + 1} \hspace{2.5cm}\ldots W_{n+1} = Y - (n + 1)
     \end{split}
   \end{equation*}

   Por lo tanto se cumple el paso inductivo y por el principio de
   inducción $Z_n = X + nY = X + Y(Y - W_n) = X + Y^2 - YW_n$.

   \textbf{Este procedimiento lo que hace es calcular $X + Y^2$, en la
     iteración 0, Z vale X y cuando la iteración n = Y,  la rutina
     regresa $Z = X + Y^2$, terminando el ciclo y la ejecución}
  \end{proof}

  \begin{algorithm}
    \caption{}
    \begin{algorithmic}
      \Procedure{DIFF}{X, Y; Z}
      \State $Z \leftarrow X$
      \State $W \leftarrow Y$
      \While{W > 0}
      \State $Z \leftarrow Z - 1$
      \State $W \leftarrow W - 1$
      \EndWhile
      \State \textbf{return} Z
      \EndProcedure
    \end{algorithmic}
  \end{algorithm}

\item Demuestre si la siguiente terna de Hoare es válida:


\begin{verbatim}
{k > n} if (k < n) k := n else n:= k {n = k}
\end{verbatim}

\item Dada la siguiente función:
\begin{verbatim}
public static int calcula(int n) {
    int c = 1;
    if (n == 0 || n == 1)
        return 1;
    else
        for (int i = 2; i <= n; i++)
            c = c * i;
    return c;
}
\end{verbatim}
  Demuestre que la siguiente terna de Hoare es válida:

\begin{equation*}
\{\}  w:= calcula(n) \{w \ge 2^{n - 1}\}
\end{equation*}

\end{enumerate}

\end{document}
