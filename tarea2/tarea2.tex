\documentclass[11pt]{article}
\usepackage[utf8]{inputenc}
\usepackage[spanish]{babel}
\usepackage{amsmath}
\usepackage{amsfonts}
\usepackage{amssymb}
\usepackage{amsthm}
\usepackage{makeidx}
\usepackage{graphicx}
\usepackage{verbatim}
\usepackage{algorithm}
\usepackage{mathtools}
\usepackage{algpseudocode}
\usepackage{array, tabularx}
\usepackage[left=3.2cm,right=3.2cm,top=3.2cm,bottom=3cm]{geometry}
\usepackage{fancyhdr}
\graphicspath{ {./figures/} }
\makeatletter
\renewcommand{\ALG@name}{Algoritmo}
\makeatother
\renewcommand{\qedsymbol}{$\blacksquare$}
\DeclarePairedDelimiter\ceil{\lceil}{\rceil}
\DeclarePairedDelimiter\floor{\lfloor}{\rfloor}

\author{Miguel Angel Piña Avelino\\Cinthia Rodríguez Maya}
\date{\today}
\title{Tarea 2\\Programación Avanzanda}
\fancyhf{}
\rhead{Programación Avanzada}
\lhead{Tarea 2}
\rfoot{Página \thepage}

\begin{document}
\maketitle

\begin{enumerate}

\item Demostrar que el siguiente código es correcto:
\begin{verbatim}
{a = b} if a == b then m := a else m := b {m := a}
\end{verbatim}

\item Explicar que hacen las siguientes rutinas, encontrando el
  invariante y demostrar que es válido por inducción matemática.

  \begin{algorithm}
    \caption{}
    \begin{algorithmic}
      \Procedure{COMP}{X, Y; Z}
      \State $Z \leftarrow X$
      \State $W \leftarrow Y$
      \While{W > 0}
      \State $Z \leftarrow Z + Y$
      \State $W \leftarrow W - 1$
      \EndWhile
      \State \textbf{return} Z
      \EndProcedure
    \end{algorithmic}
  \end{algorithm}

  \begin{algorithm}
    \caption{}
    \begin{algorithmic}
      \Procedure{DIFF}{X, Y; Z}
      \State $Z \leftarrow X$
      \State $W \leftarrow Y$
      \While{W > 0}
      \State $Z \leftarrow Z - 1$
      \State $W \leftarrow W - 1$
      \EndWhile
      \State \textbf{return} Z
      \EndProcedure
    \end{algorithmic}
  \end{algorithm}

\item Demuestre si la siguiente terna de Hoare es válida:


\begin{verbatim}
{k > n} if (k < n) k := n else n:= k {n = k}
\end{verbatim}

\item Dada la siguiente función:
\begin{verbatim}
public static int calcula(int n) {
    int c = 1;
    if (n == 0 || n == 1)
        return 1;
    else
        for (int i = 2; i <= n; i++)
            c = c * i;
    return c;
}
\end{verbatim}
  Demuestre que la siguiente terna de Hoare es válida:

\begin{equation*}
\{\}  w:= calcula(n) \{w \ge 2^{n - 1}\}
\end{equation*}

\end{enumerate}

\end{document}
